\chapter*{Introduzione}
\chaptermark{Introduzione}
\addcontentsline{toc}{chapter}{Introduzione}
Negli ultimi decenni, la crescente richiesta di servizi e applicazioni online ha portato ad un'esplosione dello sviluppo
delle tecnologie web. Questo cambiamento è stato reso possibile dall'aumento dell'ubiquità dei dispositivi connessi a
Internet, che rendono l'accesso ai servizi web sempre più immediato, semplice e alla portata di tutti. Di conseguenza,
le tecnologie e le applicazioni che garantiscono questi servizi devono essere in grado di rispondere in modo efficiente
a volumi di dati e utenti sempre maggiori. In questo scenario, il cloud computing rappresenta un elemento essenziale per
la gestione di applicazioni distribuite, ovvero applicazioni suddivise in più componenti spesso eterogenei, sia per
composizione che locazione fisica. 

Il cloud computing è un modello di erogazione di servizi che consente di gestire l'infrastruttura informatica necessaria
per rendere disponibili applicazioni, dati e servizi online in modo rapido, efficiente e flessibile. Grazie al cloud
computing, le risorse informatiche come server, storage e software possono essere facilmente scalate per rispondere alle
esigenze delle organizzazioni e degli utenti finali, consentendo un accesso sicuro e veloce ai servizi e alle
applicazioni da qualsiasi dispositivo connesso a Internet. La rapidità e velocità con cui il cloud computing eroga
questi servizi è data dalle tecnologie che lo compongono. \\
In questo contesto andremo a discutere di una emergente tecnologia, chiamata \textbf{WASI (WebAssembly System
Interface)}\footnote{\url{https://wasi.dev/}} e come questa possa essere considerata la terza ondata del cloud
computing\cite{cloudcomputing-thirdwave}. Grazie a WASI, è possibile eseguire applicazioni in un ambiente isolato e
sicuro, senza la necessità di dover conoscere il sistema operativo sottostante. È stata progettata per essere altamente
portabile, consentendo alle applicazioni di essere eseguite in modo efficiente su qualsiasi piattaforma. \\
Nasce e si sviluppa sopra ad una tecnologia già esistente: \textbf{WebAssembly} (o \textbf{Wasm}). Quest'ultima è
un'innovativa tecnologia nata con l'obiettivo di migliorare le prestazioni delle applicazioni web \textbf{sul browser}.
È stata progettata con l'intento di superare le limitazioni poste da Javascript. In particolare, si propone di essere
veloce, efficiente e portabile, oltre che retro-compatibile con le tecnologie già esistenti. Va notato che WebAssembly
non è pensato per sostituire JavaScript, ma piuttosto per migliorare le aree in cui quest'ultimo presenta alcune lacune:
come il rendering 3D, il video editing, giochi in-browser e così via. \\
WASI eredita tutte queste caratteristiche da Wasm e le utilizza per lo sviluppo di applicazioni \textbf{al di fuori} dei
browser. \\\\
Di seguito andremo ad approfondire WASI ed esporremo come rappresenti una tecnologia estremamente promettente per il
futuro nell'ambito del cloud computing. Si partirà affrontando l'argomento da un punto di vista generale, andando a
definire le motivazioni storiche che hanno portato alla sua ideazione, il suo funzionamento e lo stato dell'arte della
tecnologia. Lo si metterà a confronto con le soluzioni esistenti, in particolare affrontando le sue principali
somiglianze e differenze con il modello a container, estremamente popolare al giorno d'oggi e con il suo predecessore,
le macchine virtuali. Si studieranno le principali caratteristiche ereditate  da Wasm ponendo particolare attenzione
alla loro implementazione fuori dal browser da un punto di vista tecnico. Tra queste caratteristiche ritroviamo:
\begin{itemize}
    \item \textbf{efficienza}, dato il suo formato binario di piccole dimensioni, simile al linguaggio macchina
    \item \textbf{portabilità}, grazie alla possibilità di essere eseguito su molteplici sistemi e architetture allo
    stesso modo mediante applicazioni dette runtime simili, per funzionamento, alla JVM di Java.
    \item \textbf{interoperabilità}, grazie al fatto che non è strettamente legato ad alcun linguaggio di programmazione
    specifico
    \item \textbf{sicurezza}, in quanto eseguito in ambiente sandbox secondo il modello capability-based
\end{itemize}

La spiegazione teorica sul funzionamento e sui vantaggi sfocerà in un'implementazione di un piccolo prototipo scritto in
linguaggio Rust basato su un'architettura a microservizi REST API in cui si andranno a mettere in pratica i concetti
visti in precedenza. Si andrà a valutare il deployment dell'applicazione e ad effettuare benchmark per valutarne i tempi
di risposta a livelli di carico differenti per simularne un caso d'uso reale. I risultati ottenuti saranno messi a
confronto con un'applicazione che propone le stesse funzionalità ma implementata con una diversa tecnologia.