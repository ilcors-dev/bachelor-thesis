\chapter*{Conclusione}
\chaptermark{Conclusione}
\addcontentsline{toc}{chapter}{Conclusione}

In un mondo sempre più connesso in cui le applicazioni distribuite sono in costante crescita, la sicurezza, la
portabilità e l'efficienza sono fattori cruciali. WASI (WebAssembly System Interface) si presenta come uno standard
emergente del W3C per eseguire applicazioni WebAssembly in ambiente distribuito che potrebbe diventare un tassello
importante nell'industria informatica, tanto che è definito da alcuni come la terza ondata del cloud computing. Nel
corso dell'elaborato, è stata condotta un'analisi delle sue caratteristiche, dei suoi vantaggi e delle problematiche
riscontrate.  

Nel primo capitolo del presente lavoro si è collocato WASI all'interno dell'ecosistema delle tecnologie web,
evidenziando il suo stretto legame con WebAssembly. A tal proposito, si è evidenziata la sua somiglianza al
linguaggio Java, poiché entrambi condividono il concetto di WORA (Write Once Run Anywhere), e si sono messi in risalto i
tratti distintivi che li differenziano. In seguito, si è approfondita l'analisi tecnica di WASI, al fine di comprendere
il suo effettivo funzionamento low-level, se ne è evidenziata la forte predisposizione alla sicurezza basata sul modello
capability-based e si è analizzata la possibile collocazione della tecnologia in ambiente distribuito, focalizzandosi in
particolare sulla sua natura di container leggero, sulle differenze con i container classici e le Virtual Machine.
L'analisi tecnica condotta ha permesso di sviluppare un Proof of Concept che ha dimostrato l'effettiva realizzabilità di
applicazioni WebAssembly in ambiente distribuito, nonostante queste siano state originariamente pensate per essere
eseguite soltanto all'interno del browser.

Il Proof of Concept è stato sviluppato utilizzando Spin, uno dei primi framework presenti nell'ecosistema, facendo uso
di WAGI, un'implementazione delle CGI per WebAssembly e strutturando l'applicazione secondo il modello a microservizi.
Se ne è dimostrata la complementarità con Docker e il processo di deployment tramite container OCI. Successivamente,
durante la fase di valutazione, si è confrontato il progetto con un'applicazione simile sviluppata in NodeJS. Questo
confronto ha evidenziato la natura ancora acerba di WASI rispetto alle soluzioni esistenti in termini di prestazioni ma
ne ha evidenziato la grande differenza per quanto riguarda la portabilità e la facilità di deployment. Nonostante i
risultati del PoC non siano stati completamente soddisfacenti, la valutazione di WASI rimane positiva in quanto la
tecnologia presenta un grande potenziale che potrebbe essere migliorato ulteriormente con lo sviluppo futuro. È
importante sottolineare che WASI è ancora in fase di sviluppo, molte funzionalità cruciali come il networking, il
supporto ai thread e così via sono ancora in fase di discussione e, in quanto standard, sta impiegando del tempo per
essere sviluppato. Il fatto di essere uno standard W3C ne garantisce la stabilità a lungo termine ma sta obbligando
alcune realtà esistenti a implementare soluzioni custom per conseguire i risultati desiderati nel breve termine.

In conclusione, la sicurezza garantita dal modello capability based, la portabilità grazie alla natura basata su
WebAssembly e l'efficienza grazie all'uso dei container leggeri sono i principali vantaggi di WASI. Questi fattori,
insieme alla possibilità di eseguire codice in modo isolato e sicuro all'interno di una sandbox, possono rendere WASI
una valida opportunità futura per lo sviluppo di applicazioni in ambiente distribuito.