\chapter{Conclusione}
\label{chap:conclusion}
La conclusione della tesi fornisce un giudizio moderatamente positivo sulla tecnologia WASI studiata nel documento.
Nel corso dell'elaborato sono state analizzate le caratteristiche, i vantaggi e le problematiche teoriche relative a
questa tecnologia emergente, culminando nella presentazione di un Proof of Concept che ha dimostrato l'effettivo
funzionamento di Wasm al di fuori del browser.

Nonostante i risultati del PoC non siano stati del tutto soddisfacenti, questo non ha minato la valutazione positiva di WASI. Si è sottolineata la sua grande potenzialità e la possibilità che, data la sua attuale immaturità, i risultati ottenuti possano migliorare notevolmente con lo sviluppo futuro della tecnologia. È importante sottolineare anche che l'ecosistema di WASI è ancora relativamente piccolo, ma sta guadagnando sempre più attenzione tra le aziende del settore cloud, dimostrando ulteriormente il suo valore.

Le caratteristiche principali di WASI:
\begin{itemize}
    \item la sicurezza garantita dal modello capability based;
    \item la portabilità grazie alla sua natura basata su WebAssembly;
    \item l'efficienza grazie all'uso dei container leggeri;
\end{itemize}

sono state evidenziate come i principali vantaggi della tecnologia. Questi fattori, insieme alla possibilità di eseguire
codice in modo isolato e sicuro, rendono WASI una valida alternativa alle soluzioni esistenti, spesso basate su
container virtualizzati.

Inoltre, è importante notare che l'utilizzo di WASI potrebbe offrire significative opportunità per lo sviluppo di
applicazioni distribuite e scalabili, grazie alla sua portabilità e alla sua capacità di eseguire codice in ambienti
diversi.